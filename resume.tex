% !TEX program = xelatex

\documentclass[10pt]{resume}
%\usepackage{tabu}
%\usepackage{multirow}
%\usepackage{progressbar}
\usepackage{zh_CN-Adobefonts_external} % Simplified Chinese Support using external fonts (./fonts/zh_CN-Adobe/)
%\usepackage{zh_CN-Adobefonts_internal} % Simplified Chinese Support using system fonts
\usepackage{linespacing_fix} % disable extra space before next section
\usepackage{cite}

\begin{document}
\pagenumbering{gobble} % suppress displaying page number

\name{金航羽}

\basicInfo{
  \email{735525714@qq.com} \textperiodcentered\ 
}
\section{\faGraduationCap\ 教育经历}
\datedsubsection{\textbf{中山大学}, Guangzhou, China}{2016至今}
财务管理学士, GPA: 3.7/4

\textbf{主修课程}: 会计, 期权、期货及其他衍生品, 财务报表分析, 高等数学,
线性代数, 概率论与数理统计, 管理信息系统, 审计学, 货币银行学, 计量经济学

\section{\faUsers\ 活动实践}
\datedsubsection{\textbf{中山大学校团工委社团部}}{2016/09 - 2017/06}

\textbf{经历}: 参与组织策划中山大学校园社团系列活动百团大战及社团文化节活动,主要负责舞台策划。在组内独立负
责外校联谊访问及部门赞助事宜。参与运营社团部公众号。

\datedsubsection{\textbf{中山大学吉他协会}}{2016/09 - 2017/01}

\section{\faCogs\ 实习经历}
\datedsubsection{\textbf{东亚银行广州分行}\textit{(企业银行部)}}{2018/07 - 2018/09}

\textbf{经历}: 核实企业授信条件、撰写信贷报告,开发企业授信及中间业务产品,分析整理客户数据并进行业务营销等。

\datedsubsection{\textbf{太平洋证券总部}\textit{(资产管理部)}}{2019/01 - 2019/04}

\textbf{经历}: 固定收益及权益理财管理服务,投融资和市值管理服务,承销管理资产证券化,设计债券资管产品及衍生
对冲产品,实现一级半市场的定增等。

% Reference Test
%\datedsubsection{\textbf{Paper Title\cite{zaharia2012resilient}}}{May. 2015}
%An xxx optimized for xxx\cite{verma2015large}
%\begin{itemize}
%  \item main contribution
%\end{itemize}

\section{\faFlask\ 学术研究}
\textbf{建模报告}: 规划特斯拉电动化普及时间及规模、优化机场安检站的乘客吞吐量、优化公交路线查询算法、期权BS定价模型优化

\datedsubsection{\textbf{科研能力}: 参与学校博士后科研小组}{2018/03至今}

\textit{研究方向}: 博弈论与机制设计,已读相关国内外论文数十篇,已完成一篇论文。

\datedsubsection{MSCI}{2019/02 - 2019/04}

使用Python,\textit{M}atlab和其他工具研究随机变量的分布,期权定价,计算VaR和ES,优化定价模型和评估企业投资。

\section{\faHeartO\ 获奖情况}
中山大学优秀学生奖学金

\datedline{Bridge+第二届全国青年商战模拟大赛二等奖}{2016/10}
\datedline{“数创杯”全国大学生数学建模挑战赛二等奖}{2017/11}
\datedline{美国大学生数学建模竞赛H奖}{2018/02}

\section{\faInfo\ 专业能力}
\begin{itemize}[parsep=0.5ex]
  \item \textbf{数理分析能力}: 有一定数理背景,逻辑清晰,大学所修数学类课程均在95分以上,且正在辅修数学学院相关课程。有数学建模经历,能够基本运用\textit{M}atlab,Python和\LaTeX
  \item \textbf{数据分析能力}: 能有效搜集数据,并用数据库整合数据
  \item  \textbf{组织策划能力}:策划两场校级舞台活动、赞助活动,组织与广州高校的外访联谊活动,具有高效沟通和危机处理能力。 
\end{itemize}

%% Reference
%\newpage
%\bibliographystyle{IEEETran}
%\bibliography{mycite}
\end{document}
